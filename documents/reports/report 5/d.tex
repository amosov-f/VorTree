
    Проведем анализ сложности, считая, что $d$ велико, а $m$ --- по прежнему небольшая константа. 
    
    \paragraph{Сложность на больших размерностях\\}
        Основной факт, которым мы будем далее пользоваться --- это то, что в триангуляции Делоне на $n$ точках в $d$ мерном пространстве $n^{\lceil \frac{d}{2} \rceil}$ симплексов. Теперь число всех граней в $L_i$ равен $O(d n^{\lceil \frac{d}{2} \rceil})$.
    
        Пусть теперь мы сразу знаем выпуклую оболочку графа Делоне и пограничные симплексы. Аналогично, считаем $T(n)$,    
        \begin{enumerate}
            \item Выбор $m$ случайных точек --- $O(m) = O(1)$
            \item Разбиение всех точек по $m$ клеткам. С учетом того, что $m$ --- константа, мы можем это сделать за $O(d n f(m)) = O(d n)$, где $f(m)$ --- некоторая малая функция типа $\log m$ и т.п.
            \item Построение всех $D_i$ --- $m T(\frac{n}{m})$
            \item Запуск всех dfs --- $O(d n^{\lceil \frac{d}{2} \rceil}) + O(m g(\frac{n}{m}) \log n)$, где $g(n)$ --- ориентировочное число граней выпуклой оболочки случайного множества из $n$ точек. Для $d$--мерного случая $g(n) = O(n^{\lceil \frac{d - 1}{2} \rceil})$. $\log n$ вылезает из--за поиска точки в круге при проверке треугольника на <<хорошесть>>, если у нас уже построен поисковый индекс на $P$ (или его частях). Тем самым, сложность получается такой, $O(d n^{\lceil \frac{d}{2} \rceil}) + O(n^{\lceil \frac{d - 1}{2} \rceil} \log n) = O(d n^{\lceil \frac{d}{2} \rceil})$.
            \item Конструирование $V$ --- $O(n)$.
            \item Построение $G$ --- $O(T(|V|)) = O(T(m g\left(\frac{n}{m}\right))) = O(T(d n^\frac{d - 1}{d})) = o(T(n))$, в случае рекурсивного вызова. Опыт показывает, что если $n$ достаточно велико, то вызов внешнего алгоритма будет происходить только на задачах малых, по сравнению с $n$, размеров, поэтому его временем работы можно пренебречь.
            \item Нахождение нужных ребер --- $O(|G|) = O(|V|^2) = O \left( d^2 n^{\left( \frac{d - 1}{d} \right)^2} \right)$
            \item Вставка ребер (удаление уже было произведено на этапе нахождения плохих треугольников) --- $O(|G|) = O\left(d^2 n^{\left( \frac{d - 1}{d} \right)^2} \right)$
    \end{enumerate}  
    
    Ясно, что теперь 
    $$
        T(n) = m T \left( \frac{n}{m} \right) + O(d n^{\lceil \frac{d}{2} \rceil})
    $$      
    По теореме,
    $$
        T(n) = O(d n^{\lceil \frac{d}{2} \rceil})
    $$
    
    Сложность не может быть меньше, чем число симплексов, поэтому она такая большая.