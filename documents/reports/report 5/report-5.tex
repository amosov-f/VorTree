\documentclass{article}
\usepackage[utf8]{inputenc}
\usepackage[russian]{babel}
\usepackage{graphics}
\usepackage{amsfonts}
\usepackage{amssymb}

\ifx\pdfoutput\undefined
\usepackage{graphicx}
\else
\usepackage[pdftex]{graphicx}
\fi

\hoffset -2.0cm	
\voffset -3.0cm
\textheight 23.5cm 
\textwidth 17.0cm

\title{\bf Отчет \No 5}
\author{Амосов Федор}

\begin{document}
	\maketitle

    \paragraph{Алгоритм\\}
	Итак, вспомним наш алгоритм построения графа Делоне $D$ на наборе точек $P$ через склеивание меньших подграфов.
    \begin{enumerate}
        \item Пусть $S_i$ --- случайный поднабор точек из $P$ размера $m$;
        \item Пусть $P_i$ --- разбиение точек $P$ по клеткам Вороного $C_i$ набора точек $S_i$ (каждое $P_i$ --- это множество точек);
        \item Построим граф Делоне $D_i$ на каждом множестве $P_i$ рекурсивным вызовом этого алгоритма. Пусть каждый наш граф Делоне хранит помимо построенного графа еще и $L_i$ --- список смежных каждой грани симплексов (треугольников). Далее нам будет полезен тот факт, что размер всех $L_i$ равен $O(n)$.  
        \item Проделаем с каждым $D_i$ следующую операцию. Возьмем все граничные треугольники (находятся из $L_i$). Запустим dfs, описанный в предыдущем отчете, на графе соседних треугольников. С помощью него мы найдем все <<плохие>> треугольники. Выбросим все плохие треугольники из $D_i$ (вместе с соответствующими ребрами и вершинами);
        \item Сконструируем множество точек $V$. Добавим в него все точки границ $Conv P_i$ (находятся из $L_i$). Так же добавим в него все вершины найденных плохих треугольников;
        \item Построим $G$ --- граф Делоне на $V$. Сделаем мы это рекурсивным вызовом этого алгоритма, или каким--нибудь {\bf другим} построителем графов Делоне, если точек в $V$ достаточно мало$^*$;
        \item Найдем в $G$ те ребра, которые либо,
            \begin{itemize}
                \item Связывают вершины разных $D_i$;
                \item Были удалены в ходе уничтожения плохих треугольников.
            \end{itemize}
        \item Получим итоговый граф Делоне $D$ вставкой этих ребер в объединение $D_i$.
    \end{enumerate}    
    
    На сей момент, корректность этого алгоритма была <<проверена>> только многочисленными экспериментами. Но сейчас нам будет интересен другой вопрос. Сколько этот алгоритм работает (в количестве операций)?
    
    \paragraph{Сложность на малых размерностях\\}
    
    Итак, пусть $T(n)$ --- время работы этого алгоритма на наборе из $n$ $d$--мерных точек, где $d$ --- небольшая константа. Составим рекуррентное соотношение на $T(n)$. Предположения, в которых мы будем это делать,
    \begin{itemize}
        \item Все $Conv P_i$ имеют высокую выпуклость (не выстраиваются в линии)
        \item Все $P_i$ имеют похожие размеры
    \end{itemize}    	
    Добиться этого можно взяв $m$ (число множеств $P_i$) достаточно большим. Будем считать $m$ константой.
    
    Итак, из чего складывается $T(n)$,
    \begin{enumerate}
        \item Выбор $m$ случайных точек --- $O(m) = O(1)$
        \item Разбиение всех точек по $m$ клеткам. С учетом того, что $m$ --- константа, мы можем это сделать за $O(n f(m)) = O(n)$, где $f(m)$ --- некоторая малая функция типа $\log m$ и т.п.
        \item Построение всех $D_i$ --- $m T(\frac{n}{m})$
        \item Нахождение всех граничных треугольников (из $L_i$) + запуск всех dfs --- $O(n) + O(m g(\frac{n}{m}) \log n)$, где $g(n)$ --- ориентировочное число точек границы выпуклой оболочки случайного множества из $n$ точек. Для $d$--мерного случая $g(n) = O(d n^{\frac{d - 1}{d}}) = O(n^{\frac{d - 1}{d}})$. $\log n$ вылезает из--за поиска точки в круге при проверке треугольника на <<хорошесть>>, если у нас уже построен поисковый индекс на $P$ (или его частях). Тем самым, сложность получается такой, $O(n) + O(d n^{\frac{d - 1}{d}} \log n) = O(n)$.
        \item Конструирование $V$ --- $O(n)$.
        \item Построение $G$ --- $O(T(|V|)) = O(T(m g\left(\frac{n}{m}\right))) = O(T(n^\frac{d - 1}{d})) = o(T(n))$, в случае рекурсивного вызова. Опыт показывает, что если $n$ достаточно велико, то вызов внешнего алгоритма будет происходить только на задачах малых, по сравнению с $n$, размеров, поэтому его временем работы можно пренебречь.
        \item Нахождение нужных ребер --- $O(|G|) = [\textrm{d не большое}] = O(|V|) = O(n^\frac{d - 1}{d})$
        \item Вставка ребер (удаление уже было произведено на этапе нахождения плохих треугольников) --- $O(|G|) = O(n^\frac{d - 1}{d})$
    \end{enumerate}
    
    Итого, 
    $$
        T(n) = O(1) + O(n) + m T\left( \frac{n}{m}  \right) + O(n) + O(n) +o(T(n)) + O(n^\frac{d - 1}{d}) + O(n^\frac{d - 1}{d})
    $$    
    $$
        T(n) = m T \left( \frac{n}{m} \right) + O(n)
    $$
	
	Вспомним основную теорему о рекуррентном соотношении.
	
	\paragraph{Теорема\\}
	    Если 
	    $$
	        T(n) = a T \left( \frac{n}{b} \right) + O(n^c)
	    $$
	    То
	    $$
	        T(n) = \left\{
            	    \begin{array}{c}
				    O(n^c), ~~~ c > \log_b a    \\
				    O(n^c \log n), ~~~ c = \log_b a    \\
				    O(n^{\log_b a}), ~~~ c < \log_b a
			    \end{array}
		    \right.		
	    $$
	    
	Воспользуемся ей для нашего случая. Итого получается, 
	$$
	    T(n) = O(n \log n)	
    $$
    И это в том предположении, что $m$ и $d$ достаточно малые. Тем самым, мы разработали алгоритм, который на малых размерностях и на достаточно большом числе точек работает за $O(n \log n)$. Анализ сложности при больших $d$ будет позже. 
    
    \paragraph{Открытые задачи\\}
        Обозначения те же, что и в алгоритме. Будем рассматривать те $D_i$, которые были до пункта $4$.
        \begin{enumerate}
            \item Рассмотрим следующий неориентированный граф $T$ на $\cup_{i = 1}^m D_i$. Его вершинами будут треугольники и еще одна выделенная вершина $A$. Между двумя треугольниками будет ребро, если они имеют общую грань. Между треугольником и $A$ будет ребро, если треугольник будет находиться на границе некоторого $D_i$. Рассмотрим подграф $T'$, индуцированный на плохие треугольники и на вершину $A$. Доказать, что $T'$ связен.
            \item Рассмотрим граф $G'$ как $D$ без ребер всех $D_i$. Доказать, что множество ребер $G'$ совпадает с множеством ребер $G$, которые соединяют разные $D_i$. 
            \item Пусть $O$ --- центр описанной сферы около некоторого плохого треугольника. Пусть $O$ лежит в клетке Вороного $C_i$. Доказать, или опровергнуть то, что ближайшая к $O$ точка из $P$ принадлежит либо $C_i$, либо клетке, соседней к $C_i$.  
            \item Найти асимптотику числа граней выпуклой оболочки случайного набора точек в многомерном пространстве.             
            
        \end{enumerate}           
    
        
    
	
\end{document}
	
    	